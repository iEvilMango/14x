\documentclass{article}
\usepackage[utf8]{inputenc}

\title{Python and APIs}
\author{Joey Wunderlich}
\date{November 2016}

\begin{document}
\maketitle
\section{Basic Python (3.5) Cheatsheet}
\textit{In addition to basic python features, we'll be using the 3rd party requests library \\available at http://docs.python-requests.org/en/master/}

\subsection{Data Structures}
\subsubsection{Dictionaries (java equivalent: Maps)}
Initialization:
\begin{verbatim}
    dict1 = {"a" : 5, "b" : 10, "c" : 15}
    ## Can also use 
    ## dict2 = dict(a = 5, b = 10, c = 15)
    ## but that breaks easily; what if you want the key "a=b"?
\end{verbatim}
Put:
\begin{verbatim}
    dict1["d"] = 20
    ## Contents of dict1 : {'a': 5, 'd': 20, 'c': 15, 'b': 10}
\end{verbatim}
Get:
\begin{verbatim}
    print(dict1["d"])
    ## Prints the letter d to the console. 
\end{verbatim}
keySet:
\begin{verbatim}
    print(dict1.keys())
    ## prints a list of the keys:
    ## dict_keys(['a', 'd', 'c', 'b'])
\end{verbatim}

\subsubsection{Lists (lists)}
Initialization:
\begin{verbatim}
    list1 = ["hah", "funny", "ponies"]
\end{verbatim}
add:
\begin{verbatim}
    list1.insert(0, "first!!")
    ## resulting list = ['first!!', 'hah', 'funny', 'ponies']
    ## also available: append(value) to add to end of list.
\end{verbatim}
get:
\begin{verbatim}
    list1[0] ## gets value at index 0; "first!!"
    list1[-1]
    ## gets first value from the end (i.e. last value); 'ponies'
    list1[1:3] ## Gets elements between 1(inclusive) and 3(exclusive);
    ## ['hah', 'funny']. To get all elements, just use list1[:].
\end{verbatim}
\subsubsection{Other common data structures \&\& Miscellany}
Because we only have time for a basic overview, we're missing out on a few of the other structures. Some to look up if you're interested: sets and tuples (effectively, immutable arrays)

\subsection{Features}
\subsubsection{Consistency}
Want the size of a data structure `data`? It should always be
\begin{verbatim}
    len(data)
\end{verbatim}
Python thrives on consistency; veteran programmers in Python insist on the use of `pythonic` code; that is, code that follows the idioms created for python. Visiting https://www.python.org/dev/peps/pep-0020/ will help you get an idea of what they go for.\\\\
Helpful in this are `magic methods` - methods that overwrite behavior. Google is helpful in this regard; some common ones are \_\_repr\_\_ for the string representation of a value (as a programmer would want it), \_\_len\_\_ for overriding the len(...) function, \_\_init\_\_ for constructors and many more\\
(http://www.python-course.eu/python3\_magic\_methods.php has a nice, if non-exhaustive list to start with)

\subsubsection{Handy behavior}
Do you want to switch two variables? In python, it's as simple as
\begin{verbatim}
    a = 7
    b = 42
    a, b = b, a
    ## b == 7, a == 42
\end{verbatim}

Want to pass around functions like nobodies business? Because they're variables in python, you can do stuff like
\begin{verbatim}
    def addOne(num):
        return num + 1
    
    def doTwice(func, num):
        return func(func(num))
    
    addOne(5) ## returns 6
    doTwice(addOne, 5) ## returns 7
    
    funcMap = {"add one" : addOne, "do twice" : doTwice}
    funcMap["add one"](5) ## returns 6
\end{verbatim}

Python is white space based; as you can see above, no curly braces to surround methods, just tab in and stay there until you're done.\\

Last thing for now (it's also only `pretty much` right, but for our purposes is good enough): Most implementations of python are closer to being \textit{interpreted}, rather than \textit{compiled}; that means that we can run the code we write line by line -- even in the terminal -- without having to compile and run an entire program at once. (Just google `is python interpreted or compiled` to see endless responses on this.)\\
One more resource if you want to read a bit more about python:\\
http://docs.python-guide.org/en/latest/writing/style/

\pagebreak


\section{APIs using Python!}
\textit{We'll be using the politifact and giphy APIs to start off; more can be found at \\http://www.programmableweb.com/apis/directory}
\subsection{Getting started with the politifact API}
\textit{Documentation available at\\http://static.politifact.com/api/doc.html}
\begin{verbatim}
# import the two libraries we'll be using to make HTTP requests
import json, requests

# Define a base url to start from for all HTTP requests
base_url = "http://www.politifact.com/api/"
# Extended URL for the portion that we'll be calling from.
extended_url = base_url + "statements/truth-o-meter/json/"
# List of parameters to be used in our request
input_params = dict(
    n = 5
)

# Make the HTTP request using requests library
response = requests.get(url=extended_url, params=input_params)
## Here, response.text will contain the text obtained from the request

# Use JSON library to make that a little more usable format.
data = json.loads(response.text)
\end{verbatim}

So what do we have now? Let's dig into the data and check. (\textgreater\textgreater used to identify response)
\begin{verbatim}
print(data) 
# There's a lot of stuff their, but it looks like it's some sort of list?
>> Blob
print(data[0]) # Okay, that looks like a map. Let's call get the keys
>> Smaller blob
print(data[0].keys) #
>> dict_keys(['ruling_date', 'ruling_headline', 'target', 'statement', ...])
# Okay, so it's a list of maps. Let's get the first headline then.
print(data[0]["ruling_headline"])
>> `a headline` (it won't be up to date anyways)
## Hey, we got something usable!
\end{verbatim}

\pagebreak

\subsection{Using giphy to get some gifs}
\textit{Documentation available at\\http://api.giphy.com/}

\begin{verbatim}
import json, requests

base_url = "http://api.giphy.com/"
extended_url = base_url + "v1/gifs/search"

## Public api `beta` key from their github;
## many websites will use private keys to authenticate requests
public_api_key = "dc6zaTOxFJmzC"

# parameters; more this time. We need to pass in the API key,
# a query q to search for, and optionally a limit to the number of
# responses and a rating limit
my_params = dict(
            api_key = public_api_key,
            q = "cute cats",
            limit = 5,
            rating = "pg-13"
)
# The built up HTTP request will look like the following:
# http://api.giphy.com/v1/gifs/search?api_key
# =dc6zaTOxFJmzC&q=cute+cats&limit=5&rating=pg-13

response = requests.get(url=extended_url, params=my_params)
data = json.loads(response.text)
\end{verbatim}
Now, let's look at it!


\begin{verbatim}
data[0] # Key error: must not be a list this time?
# Looking at it in firefox, we can see that it actually
# is a map of lists of maps this time; we want the key "data", and then
# the first element of that
>> Traceback (most recent call last):
    File "<stdin>", line 1, in <module>
    KeyError: 0
  
data["data"][0]
# Okay, got a lot of stuff here... looks like we want the bitly_url!

data["data"][0]["bitly_url"]
>> 'http://gph.is/1XYsr1n' # SUCCESS!
\end{verbatim}



\end{document}